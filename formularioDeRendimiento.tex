\documentclass{article}

\usepackage[spanish]{babel}
\usepackage{amsmath}
\usepackage{geometry}
\usepackage{graphicx}
\usepackage{pgfplots}
\usepackage{url}


\title{Formulario Capitulo 1}
\author{nombre: Martin Natera C.D: 30.445.341}
\geometry{margin=2cm}

\begin{document}
	\maketitle
	\begin{table}[h]
		\centering
		\begin{tabular}{p{8cm}p{8cm}}
			\hline
			                 \textbf{  Prestaciones:}        \\
			\hline \\
			$prestaciones_{x}=\frac{1}{tiempo \ de \ ejercucion_{x}}$	& $prestaciones_{x} > prestaciones_{y} $    \\
			\\
			$\frac{1}{tiempo \ de \ ejercucion_{x}} > \frac{1}{tiempo \ de \ ejercucion_{y}} $& $tiempo \ de \ ejercucion_{y} > tiempo \ de \ ejercucion_{x}$     \\
			\\
			$\frac{prestaciones_{x}}{prestaciones_{y}}= n$ &$\frac{tiempo \ de \ ejecucion_{y}}{tiempo \ de \ ejecucion_{x}}= n$      \\
			\\
			\hline \\
			\textbf{Prestaciones de la CPU:} \\
			\hline \\
			 Tiempo de ejecución de CPU=  Ciclos de reloj de la CPU $\times$ Tiempo del ciclo del reloj &  Tiempo de ejecución de CPU=$\frac{Ciclos \ de \ reloj \ de \ la \ CPU}{ Frecuencia de reloj}$ \\ \\
			\hline \\
			Tiempo de $CPU_{X}$=$\frac{Ciclos \ de \ reloj \ de \ CPU_{X}}{Frecuencia \ de \ reloj_{Y}}$ & \\
			\hline \\
		\textbf{	Prestaciones de las instrucciones:} \\
			\hline \\
			Ciclos de reloj de CPU = Instrucciones de un programa $\times$ Media de ciclos 
			por instrucción & \\
			\hline \\
			prestaciones de la CPU: \\
			\hline \\
			Tiempo de ejecución= Número de instrucciones$\times CPI \times$Tiempo de ciclo & Tiempo de ejecución=$\frac{Numero \ de \ instrucciones}{Frecuencia \ de \ reloj}$ \\
			\hline \\
			CPI=$\frac{Ciclos \ de \ reloj \ de \ CPU}{Numero \ de \ instrucciones}$&  Ciclos de reloj de la CPU=$\sum_{i=1}^{n}(CPI_{i}\times C_{i})$  \\ \\
			\hline \\
		\end{tabular}
	\end{table}
\end{document}